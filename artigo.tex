\documentclass{pssbmac}

%%%%%%%%%%%%%%%%%%%%%%%%%%%%%%%%%%%%%%%%%%%%%%%%%%%%%%%%%%%%%%%%%%%%%%%%
%% POR FAVOR, NÃO FAÇA MUDANÇAS NESSE PADRÃO QUE ACARRETEM  EM
%% ALTERAÇÃO NA FORMATAÇÃO FINAL DO TEXTO
%%%%%%%%%%%%%%%%%%%%%%%%%%%%%%%%%%%%%%%%%%%%%%%%%%%%%%%%%%%%%%%%%%%%%%%%

%%%%%%%%%%%%%%%%%%%%%%%%%%%%%%%%%%%%%%%%%%%%%%%%%%%%%%%%%%%%%%%%%%%%%%%%
% POR FAVOR, ESCOLHA CONFORME O CASO
%%%%%%%%%%%%%%%%%%%%%%%%%%%%%%%%%%%%%%%%%%%%%%%%%%%%%%%%%%%%%%%%%%%%%%%%
%\usepackage[english]{babel} % texto em Inglês

%\usepackage[latin1]{inputenc} % acentuação em Português ISO-8859-1
\usepackage[utf8]{inputenc} % acentuação em Português UTF-8
%%%%%%%%%%%%%%%%%%%%%%%%%%%%%%%%%%%%%%%%%%%%%%%%%%%%%%%%%%%%%%%%%%%%%%%%


%%%%%%%%%%%%%%%%%%%%%%%%%%%%%%%%%%%%%%%%%%%%%%%%%%%%%%%%%%%%%%%%%%%%%%%%
%% POR FAVOR, NÃO ALTERAR
%%%%%%%%%%%%%%%%%%%%%%%%%%%%%%%%%%%%%%%%%%%%%%%%%%%%%%%%%%%%%%%%%%%%%%%%
\usepackage[UTF8]{ctex}  % 主要宏包
\usepackage[T1]{fontenc}
\usepackage{float}
\usepackage{graphics}
\usepackage{graphicx}
\usepackage{epsfig}
\usepackage{indentfirst}
\usepackage{amsmath, amsfonts, amssymb, amsthm, mathtools}
\usepackage{url}
\usepackage{csquotes}
% Ambientes pré-definidos
\newtheorem{theorem}{Theorem}[section]
\newtheorem{lemma}{Lemma}[section]
\newtheorem{proposition}{Proposition}[section]
\newtheorem{definition}{Definition}[section]
\newtheorem{remark}{Remark}[section]
\newtheorem{corollary}{Corollary}[section]
\newtheorem{teorema}{Teorema}[section]
\newtheorem{lema}{Lema}[section]
\newtheorem{prop}{Proposi\c{c}\~ao}[section]
\newtheorem{defi}{Defini\c{c}\~ao}[section]
\newtheorem{obs}{Observa\c{c}\~ao}[section]
\newtheorem{cor}{Corol\'ario}[section]

% ref bibliográficas
% \usepackage[backend=biber, style=numeric-comp, maxnames=50]{biblatex}
% \addbibresource{refs.bib}
% \DeclareTextFontCommand{\emph}{\boldmath\bfseries}
% \DefineBibliographyStrings{brazil}{phdthesis = {Tese de doutorado}}
% \DefineBibliographyStrings{brazil}{mathesis = {Disserta\c{c}\~{a}o de mestrado}}
% \DefineBibliographyStrings{english}{mathesis = {Master dissertation}}
%%%%%%%%%%%%%%%%%%%%%%%%%%%%%%%%%%%%%%%%%%%%%%%%%%%%%%%%%%%%%%%%%%%%%%%%


\begin{document}

%%%%%%%%%%%%%%%%%%%%%%%%%%%%%%%%%%%%%%%%%%%%%%%%%%%%%%%%%%%%%%%%%%%%%%%%
% TÍTULO E AUTORAS(ES)
%%%%%%%%%%%%%%%%%%%%%%%%%%%%%%%%%%%%%%%%%%%%%%%%%%%%%%%%%%%%%%%%%%%%%%%%

\title{《Linear Convergence Rate in Convex Setup is Possible!》一文的问题回答}

\author{
{\large 姓名学号}\\ {\large 姓名学号}\\  {\large 姓名学号}\\   
}
\criartitulo
%%%%%%%%%%%%%%%%%%%%%%%%%%%%%%%%%%%%%%%%%%%%%%%%%%%%%%%%%%%%%%%%%%%%%%%%


%%%%%%%%%%%%%%%%%%%%%%%%%%%%%%%%%%%%%%%%%%%%%%%%%%%%%%%%%%%%%%%%%%%%%%%%
% TEXTO
%%%%%%%%%%%%%%%%%%%%%%%%%%%%%%%%%%%%%%%%%%%%%%%%%%%%%%%%%%%%%%%%%%%%%%%%

\begin{abstract}
{\bf Resumo}. Este é o padrão (formato \LaTeX{} apenas) para a submissão de trabalhos da Categoria 2 do CNMAC, destinados à divulgação de pesquisas com resultados conclusivos. \emph{Nesta categoria, os trabalhos devem ser submetidos em Português ou Inglês, em forma de artigo com no mínimo 5 e no máximo 7 páginas, incluindo-se as referências bibliográficas.} Os \emph{trabalhos submetidos} que \emph{não estiverem de acordo com o formato} apresentado por esse padrão \emph{serão rejeitados} pelo Comitê Editorial do evento, sem análise do mérito científico.  

\noindent
{\bf Palavras-chave}. Instruções, \LaTeX, Trabalhos Completos, SBMAC, CNMAC  (entre 3-6 palavras-chave)
\end{abstract}

\section{问题一:论文解决的问题}


\subsection{更优秀的梯度下降法分析}

过去的论文在凸性和 $(L_0, L_1)$-smoothness 假设下的研究要么依赖于更严格的假设,如Koloskova 等研究者的成果依赖于一个额外的 $L$-smooth 条件,Takezawa 等研究者的成果要求梯度下降带 Polyak 步长;要么分析结果过于粗糙,如 Li 等研究者的分析结果依赖一个可能很大的常数;要么对于收敛速率研究不够细致,例如Gorbunov 和 Vankov 等研究者没有研究梯度下降刚开始时的线性收敛。

论文改进了现有对 $(L_0, L_1)$-GD、NGD 和 Clip-GD 在凸性和 $(L_0, L_1)$-smoothness 假设下的收敛性分析。
展示了在凸性和 $(L_0, L_1)$-smoothness 假设下,
这些方法在初始优化阶段(即当 $\|\nabla f(x^{k})\| \geq \frac{L_0}{L_1}$ 时)
无需任何额外假设即可实现线性收敛;
随着迭代接近最优解,收敛速度逐渐转为次线性。
特别地,对于 NGD,当 $\|\nabla f(x^{k})\| \geq c$ 时,
仍保持线性收敛。论文提供的 表~1 明确给出了 Clip-GD 线性收敛的条件:
其中 $\lambda_k = 1$ 对应 GD 的情形,
$\lambda_k = \frac{c}{\|\nabla f(x^{k})\|}$ 对应 NGD 的情形。

\subsection{对于 RCD 和 OrderRCD 方法的首次分析}
论文首次为随机坐标下降方法(RCD)及 OrderRCD 在凸性和
 $(L_0, L_1)$-coordinate smoothness 假设下提供了收敛性分析。
 结果表明,这两种方法同样展现出与 full-gradient 方法类似的“先线性、
 后次线性”的收敛现象。

\subsection{Extension to the strongly convex case}
此外,论文还将 $(L_0, L_1)$-GD 的分析推广至目标函数满足
 $\mu$-strongly convex 的情形,
 进一步拓展了该类非标准光滑条件下的优化理论。

\section{问题二:文章的假设条件}

\subsection{形式化表述}

\subsubsection{L-smoothness(Lipschitz 梯度)}

函数 $f: \mathbb{R}^d \to \mathbb{R}$ 被称为 L-smooth,如果其梯度 $\nabla f$ 是 L-Lipschitz 连续的,即对任意 $x, y \in \mathbb{R}^d$ 有
$$
\|\nabla f(x) - \nabla f(y)\| \leq L \|x - y\|
$$
若其有二阶梯度,那么有
$$
\|\nabla^2 f(x)\| \leq L
$$

\subsubsection{$(L_0, L_1)$-smoothness}
函数 $f$ 满足 $(L_0, L_1)$-smoothness,如果对任意 $x, y \in \mathbb{R}^d$,满足 $ \| x-y\|\leq \frac{1}{L_1} $,有
$$
\|\nabla f(x) - \nabla f(y)\| \leq (L_0 + L_1 \|\nabla f(x)\|) \|x - y\|
$$

\subsection{假设区别}
\begin{table}[H]
\caption{ {\small $(L_0, L_1)$-smoothness 假设和 L-smoothness 假设的不同}}
\centering
\begin{tabular}{ccc}
\hline
性质  & L-smoothness 假设 & \small $(L_0, L_1)$-smoothness 假设\\ \hline
梯度变化剧烈程度         & 全局固定 $L$  & 依赖当前位置梯度:$L_0 + L_1 \|\nabla f(x)\|$   \\
严格程度          & 要求梯度变化一致有界  & 允许梯度变化与当前位置相关 \\
强弱关系 & 更强 & 更弱 \\
\hline
\end{tabular}\label{tabela01}
\end{table}

\subsection{严格程度}

$(L_0, L_1)$-smoothness 是比 L-smoothness 更弱、更宽松的条件,因为其允许梯度在一定情况下变化可以较快,但是 L-smoothness 强制要求梯度变化必须保持缓慢,以下是一个详细证明。

若 $f$ 是 L-smooth,则对于 $\|\nabla f(x) - \nabla f(y)\| \leq L \|x - y\|$ 取 $L_0 = L$, $L_1 = 0$,显然满足$\|\nabla f(x) - \nabla f(y)\| \leq (L_0 + L_1 \|\nabla f(x)\|) \|x - y\|$,即 $(L_0, L_1)$-smoothness。


反之,存在函数满足 $(L_0, L_1)$-smoothness 但不满足任何 L-smoothness。例如,考虑 $f(x) = e^x$ 在 $x\in \mathbb{R}$ 时,其梯度为 $e^x$,也是 $e^x$,无全局上界,故不存在 $L$ 使得$\|\nabla f(x) - \nabla f(y)\| \leq L \|x - y\|$,因而它是非 L-smooth 的。但可验证其满足 $(0, 1)$-smoothness,因为
$$
|f'(x) - f'(y)| = |e^x - e^y| \leq e^{\max(x,y)} |x - y| \leq (\underbrace{0}_{L_0} + \underbrace{1}_{L_1} \cdot |f'(x)|) |x - y|.
$$

因此,L-smoothness $\Rightarrow$ $(L, 0)$-smoothness,但逆命题不成立。  
所以,$(L_0, L_1)$-smoothness 是比 L-smoothness 更弱、更宽松的条件。



\section{问题三:文章的不足和前景}

\subsection{假设过于严格造成的应用范围狭窄}

文中自己提到,学界和工业界大多都集中于研究更普适的非全局凸函数,但是本文的研究均是全局凸函数的情形,因此其应用范围有限。


\section{模板留档}

\subsection{表格示例}

\begin{table}[H]
\caption{ {\small Categorias dos trabalhos.}}
\centering
\begin{tabular}{ccc}
\hline
Categoria do trabalho  & Número de páginas & Tipo do trabalho\\ \hline
1          & 2  & $A$, $B$ e $C$    \\
2          & entre 5 e 7  & apenas $C$ \\
\hline
\end{tabular}\label{tabela01}
\end{table}

\begin{figure}[H]
\centering
\includegraphics[width=.7\textwidth]{ex_fig}
\caption{ {\small Exemplo de imagem. Fonte: indicar.}}
\label{figura01}
\end{figure}

%%%%%%%%%%%%%%%%%%%%%%%%%%%%%%%%%%%%%%%%%%%%%%%%%%%%%%%%%%%%%%%%%%%%%%%%
% REFS BIBLIOGRÁFICAS
% POR FAVOR, NÃO ALTERAR
%%%%%%%%%%%%%%%%%%%%%%%%%%%%%%%%%%%%%%%%%%%%%%%%%%%%%%%%%%%%%%%%%%%%%%%%
% \printbibliography
%%%%%%%%%%%%%%%%%%%%%%%%%%%%%%%%%%%%%%%%%%%%%%%%%%%%%%%%%%%%%%%%%%%%%%%%

\end{document}




