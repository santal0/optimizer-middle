\documentclass{pssbmac}

%%%%%%%%%%%%%%%%%%%%%%%%%%%%%%%%%%%%%%%%%%%%%%%%%%%%%%%%%%%%%%%%%%%%%%%%
%% POR FAVOR, NÃO FAÇA MUDANÇAS NESSE PADRÃO QUE ACARRETEM  EM
%% ALTERAÇÃO NA FORMATAÇÃO FINAL DO TEXTO
%%%%%%%%%%%%%%%%%%%%%%%%%%%%%%%%%%%%%%%%%%%%%%%%%%%%%%%%%%%%%%%%%%%%%%%%

%%%%%%%%%%%%%%%%%%%%%%%%%%%%%%%%%%%%%%%%%%%%%%%%%%%%%%%%%%%%%%%%%%%%%%%%
% POR FAVOR, ESCOLHA CONFORME O CASO
%%%%%%%%%%%%%%%%%%%%%%%%%%%%%%%%%%%%%%%%%%%%%%%%%%%%%%%%%%%%%%%%%%%%%%%%
\usepackage[brazil]{babel} % texto em Português
%\usepackage[english]{babel} % texto em Inglês

%\usepackage[latin1]{inputenc} % acentuação em Português ISO-8859-1
\usepackage[utf8]{inputenc} % acentuação em Português UTF-8
%%%%%%%%%%%%%%%%%%%%%%%%%%%%%%%%%%%%%%%%%%%%%%%%%%%%%%%%%%%%%%%%%%%%%%%%


%%%%%%%%%%%%%%%%%%%%%%%%%%%%%%%%%%%%%%%%%%%%%%%%%%%%%%%%%%%%%%%%%%%%%%%%
%% POR FAVOR, NÃO ALTERAR
%%%%%%%%%%%%%%%%%%%%%%%%%%%%%%%%%%%%%%%%%%%%%%%%%%%%%%%%%%%%%%%%%%%%%%%%
\usepackage[UTF8]{ctex}  % 主要宏包
\usepackage[T1]{fontenc}
\usepackage{float}
\usepackage{graphics}
\usepackage{graphicx}
\usepackage{epsfig}
\usepackage{indentfirst}
\usepackage{amsmath, amsfonts, amssymb, amsthm, mathtools}
\usepackage{url}
\usepackage{csquotes}
% Ambientes pré-definidos
\newtheorem{theorem}{Theorem}[section]
\newtheorem{lemma}{Lemma}[section]
\newtheorem{proposition}{Proposition}[section]
\newtheorem{definition}{Definition}[section]
\newtheorem{remark}{Remark}[section]
\newtheorem{corollary}{Corollary}[section]
\newtheorem{teorema}{Teorema}[section]
\newtheorem{lema}{Lema}[section]
\newtheorem{prop}{Proposi\c{c}\~ao}[section]
\newtheorem{defi}{Defini\c{c}\~ao}[section]
\newtheorem{obs}{Observa\c{c}\~ao}[section]
\newtheorem{cor}{Corol\'ario}[section]

% ref bibliográficas
\usepackage[backend=biber, style=numeric-comp, maxnames=50]{biblatex}
\addbibresource{refs.bib}
\DeclareTextFontCommand{\emph}{\boldmath\bfseries}
\DefineBibliographyStrings{brazil}{phdthesis = {Tese de doutorado}}
\DefineBibliographyStrings{brazil}{mathesis = {Disserta\c{c}\~{a}o de mestrado}}
\DefineBibliographyStrings{english}{mathesis = {Master dissertation}}
%%%%%%%%%%%%%%%%%%%%%%%%%%%%%%%%%%%%%%%%%%%%%%%%%%%%%%%%%%%%%%%%%%%%%%%%


\begin{document}

%%%%%%%%%%%%%%%%%%%%%%%%%%%%%%%%%%%%%%%%%%%%%%%%%%%%%%%%%%%%%%%%%%%%%%%%
% TÍTULO E AUTORAS(ES)
%%%%%%%%%%%%%%%%%%%%%%%%%%%%%%%%%%%%%%%%%%%%%%%%%%%%%%%%%%%%%%%%%%%%%%%%

\title{《Linear Convergence Rate in Convex Setup is Possible!》一文的问题回答}

\author{
{\large 姓名学号}\\ {\large 姓名学号}\\  {\large 姓名学号}\\   
}
\criartitulo
%%%%%%%%%%%%%%%%%%%%%%%%%%%%%%%%%%%%%%%%%%%%%%%%%%%%%%%%%%%%%%%%%%%%%%%%


%%%%%%%%%%%%%%%%%%%%%%%%%%%%%%%%%%%%%%%%%%%%%%%%%%%%%%%%%%%%%%%%%%%%%%%%
% TEXTO
%%%%%%%%%%%%%%%%%%%%%%%%%%%%%%%%%%%%%%%%%%%%%%%%%%%%%%%%%%%%%%%%%%%%%%%%

\begin{abstract}
{\bf Resumo}. Este é o padrão (formato \LaTeX{} apenas) para a submissão de trabalhos da Categoria 2 do CNMAC, destinados à divulgação de pesquisas com resultados conclusivos. \emph{Nesta categoria, os trabalhos devem ser submetidos em Português ou Inglês, em forma de artigo com no mínimo 5 e no máximo 7 páginas, incluindo-se as referências bibliográficas.} Os \emph{trabalhos submetidos} que \emph{não estiverem de acordo com o formato} apresentado por esse padrão \emph{serão rejeitados} pelo Comitê Editorial do evento, sem análise do mérito científico.  

\noindent
{\bf Palavras-chave}. Instruções, \LaTeX, Trabalhos Completos, SBMAC, CNMAC  (entre 3-6 palavras-chave)
\end{abstract}

\section{问题一:论文解决的问题}

本论文试图填补的缺口是:在 $(L_0, L_1)$-smoothness 的光滑条件下,为常见优化算法建立新的线性收敛理论,并解释实际机器学习训练中常见的收敛行为。具体的来说包含以下贡献:


\subsection{}


\section{问题二:文章的假设条件}

\subsection{形式化表述}

\subsubsection{L-smoothness(Lipschitz 梯度)}

函数 $f: \mathbb{R}^d \to \mathbb{R}$ 被称为 L-smooth,如果其梯度 $\nabla f$ 是 L-Lipschitz 连续的,即对任意 $x, y \in \mathbb{R}^d$ 有
$$
\|\nabla f(x) - \nabla f(y)\| \leq L \|x - y\|
$$
若其有二阶梯度,那么有
$$
\|\nabla^2 f(x)\| \leq L
$$

\subsubsection{$(L_0, L_1)$-smoothness}
函数 $f$ 满足 $(L_0, L_1)$-smoothness,如果对任意 $x, y \in \mathbb{R}^d$,满足 $ \| x-y\|\leq \frac{1}{L_1} $,有
$$
\|\nabla f(x) - \nabla f(y)\| \leq (L_0 + L_1 \|\nabla f(x)\|) \|x - y\|
$$

\subsection{假设区别}
\begin{table}[H]
\caption{ {\small $(L_0, L_1)$-smoothness 假设和 L-smoothness 假设的不同}}
\centering
\begin{tabular}{ccc}
\hline
性质  & L-smoothness 假设 & \small $(L_0, L_1)$-smoothness 假设\\ \hline
梯度变化剧烈程度         & 全局固定 $L$  & 依赖当前位置梯度:$L_0 + L_1 \|\nabla f(x)\|$   \\
严格程度          & 要求梯度变化一致有界  & 允许梯度变化与当前位置相关 \\
强弱关系 & 更强 & 更弱 \\
\hline
\end{tabular}\label{tabela01}
\end{table}

\subsection{严格程度}

$(L_0, L_1)$-smoothness 是比 L-smoothness 更弱、更宽松的条件,因为其允许梯度在一定情况下变化可以较快,但是 L-smoothness 强制要求梯度变化必须保持缓慢,以下是一个详细证明。

若 $f$ 是 L-smooth,则对于 $\|\nabla f(x) - \nabla f(y)\| \leq L \|x - y\|$ 取 $L_0 = L$, $L_1 = 0$,显然满足$\|\nabla f(x) - \nabla f(y)\| \leq (L_0 + L_1 \|\nabla f(x)\|) \|x - y\|$,即 $(L_0, L_1)$-smoothness。


反之,存在函数满足 $(L_0, L_1)$-smoothness 但不满足任何 L-smoothness。例如,考虑 $f(x) = e^x$ 在 $x\in \mathbb{R}$ 时,其梯度为 $e^x$,也是 $e^x$,无全局上界,故不存在 $L$ 使得$\|\nabla f(x) - \nabla f(y)\| \leq L \|x - y\|$,因而它是非 L-smooth 的。但可验证其满足 $(0, 1)$-smoothness,因为
$$
|f'(x) - f'(y)| = |e^x - e^y| \leq e^{\max(x,y)} |x - y| \leq (\underbrace{0}_{L_0} + \underbrace{1}_{L_1} \cdot |f'(x)|) |x - y|.
$$

因此,L-smoothness $\Rightarrow$ $(L, 0)$-smoothness,但逆命题不成立。  
所以,$(L_0, L_1)$-smoothness 是比 L-smoothness 更弱、更宽松的条件。



\section{问题三:文章的不足和前景}

\section{模板留档}


Equações inseridas no trabalho completo devem ser enumeradas sequencialmente e à direita no texto, por exemplo
\begin{equation}
\frac{\partial u}{\partial t}-\Delta u = f, \quad  \mathrm{em} \; \Omega. \label{Calor}
\end{equation}
Consulte o arquivo \verb!.tex! para mais detalhes sobre o código-fonte gerador da equação \eqref{Calor}.

As(os) autoras(es) podem inserir figuras e tabelas no artigo. Elas devem estar dispostas próximas de suas referências no texto.

\subsection{表格示例}

A inserção de tabela deve ser feita com o ambiente \verb!table!, sendo enumerada, disposta horizontalmente centralizada, próxima de sua referência no texto, e a legenda imediatamente acima dela. Por exemplo, consulte a Tabela \ref{tabela01}.

\begin{table}[H]
\caption{ {\small Categorias dos trabalhos.}}
\centering
\begin{tabular}{ccc}
\hline
Categoria do trabalho  & Número de páginas & Tipo do trabalho\\ \hline
1          & 2  & $A$, $B$ e $C$    \\
2          & entre 5 e 7  & apenas $C$ \\
\hline
\end{tabular}\label{tabela01}
\end{table}

\subsection{Inserção de Figuras}

A inserção de figura deve ser feita com o ambiente \verb!figure!, ela deve estar enumerada, disposta horizontalmente centralizada, próxima de sua referência no texto, e legenda imediatamente abaixo da mesma. \emph{Deve-se indicar/referências a fonte, mesmo quando de autoria própria.} Por exemplo, consulte a Figura \ref{figura01}.

\begin{figure}[H]
\centering
\includegraphics[width=.7\textwidth]{ex_fig}
\caption{ {\small Exemplo de imagem. Fonte: indicar.}}
\label{figura01}
\end{figure}

\section{Sobre as Referências Bibliográficas}

As referências bibliográficas devem ser inseridas conforme especificado neste padrão, as quais serão automaticamente geradas em ordem alfabética pelo sobrenome do primeiro autor. Este {\it template} fornece suporte para a inserção de referências bibliográficas com o Bib\LaTeX{}. Os dados de cada referência do trabalho devem ser adicionados no arquivo \verb+refs.bib+ e a indicação da referência no texto deve ser inserida com o comando \verb+\cite+. Seguem alguns exemplos de referências: livro \cite{Boldrini}, artigos publicados em periódicos \cite{Contiero,Cuminato}, capítulo de livro \cite{daSilva}, dissertação de mestrado \cite{Diniz}, tese de doutorado \cite{Mallet}, livro publicado dentro de uma série \cite{Gomes}, trabalho publicado em anais de eventos \cite{Santos}, {\it website} e outros \cite{CNMAC}. Por padrão, os nomes de todos os autores da obra citada aparecem na bibliografia. Para obras com mais de três autores, é também possível indicar apenas o nome do primeiro autor, seguido da expressão et al. Para implementar essa alternativa, basta remover ``\verb+,maxnames=50+'' do comando correspondente do código-fonte. Sempre que disponível forneça o DOI, ISBN ou ISSN, conforme o caso.

\section{Considerações Finais}

Esta seção é reservada às principais conclusões e considerações finais do trabalho.

\section*{Agradecimentos (opcional)}

Seção reservada aos agradecimentos dos autores, caso for pertinente. Por exemplo, agradecimento a fomentos. Se os autores optarem pela inclusão de Agradecimentos, a palavra ``(opcional)'' deve ser removida do título da seção. Esta seção não é numerada e deve ser disposta entre a última seção do corpo do texto e as Referências.


%%%%%%%%%%%%%%%%%%%%%%%%%%%%%%%%%%%%%%%%%%%%%%%%%%%%%%%%%%%%%%%%%%%%%%%%
% REFS BIBLIOGRÁFICAS
% POR FAVOR, NÃO ALTERAR
%%%%%%%%%%%%%%%%%%%%%%%%%%%%%%%%%%%%%%%%%%%%%%%%%%%%%%%%%%%%%%%%%%%%%%%%
\printbibliography
%%%%%%%%%%%%%%%%%%%%%%%%%%%%%%%%%%%%%%%%%%%%%%%%%%%%%%%%%%%%%%%%%%%%%%%%

\end{document}




